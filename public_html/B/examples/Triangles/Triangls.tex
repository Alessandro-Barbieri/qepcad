If a formula that is already quantifier free is entered into QEPCAD B,
the program produces a simple equivalent formula. As an interesting
example of formula simplification QEPCAD B, consider the following:
Figure 1 shows a triangle <em>ABC</em> with what we might
call the ``external bisector of vertex <em>B</em> with respect to <em>A</em>''.
    \scalebox{0.5}{\includegraphics{triangle.pdf}}
    <small>Triangle <em>ABC<em> with the external bisector of vertex <em>B<em>
      with respect to <em>A<em>.</small>
    \label{fig:triangle}
  \end{center}
\end{figure}
It is fairly clear from the diagram that this external bisector exists
as drawn if and only if <em>\theta > (\pi - \phi)/2<em>.  Suppose we want a
characterization in terms of the side lengths <em>a<em>, <em>b<em> and <em>c<em> rather
than the angles <em>\theta<em> and <em>\phi<em>.  Straightforward application of
the common trigonometric identities produces the
equivalent characterization that, if <em>a<em>, <em>b<em> and <em>c<em> are the side
lengths of a non-degenerate triangle, this external bisector
exists if and only if
\begin{equation}
\label{eqn:input1}
b^2 + a^2 - c^2 \leq 0 \vee c (b^2 + a^2 - c^2)^2 < a b^2 (2 a c - (c^2 + a^2 - b^2)).
\end{equation}
Of course <em>a<em>, <em>b<em> and <em>c<em> are the side lengths of a non-degenerate
triangle if and only if 
\begin{equation}
\label{eqn:assume1}
a > 0 \wedge b > 0 \wedge c > 0 \wedge a < b + c \wedge b < a + c \wedge c < a + b.
\end{equation}
The conjunction of (\ref{eqn:input1}) and (\ref{eqn:assume1}) completely characterizes the triples of
real numbers <em>(a,b,c)<em> that are side-lengths of non-degenerate
triangles for which the above described external bisector exists.  
Entering this conjunction as
input to QEPCAD B\ produces the equivalent formula
<em>c > 0 \wedge b > 0 \wedge a - b + c > 0 \wedge a - c < 0 \wedge a + b - c > 0<em>,
which is a little hard to interpret, because it's unclear which
conditions simply specify a non-degenerate triangle, and which are
specific to characterizing the existence of the external bisector.
QEPCAD B\ allows us to declare ``assumptions'', which are conditions
on the free variables that are ``assumed'' by the solution formula
produced.  A true solution formula is actually the conjunction of the
assumptions and the solution formula QEPCAD B\ produces.  Here is a
QEPCAD B\ session for this simplification problem:
{\tiny
\begin{verbatim}
=======================================================
Enter an informal description  between '[' and ']':
[ Charaterizing triangles with external bisectors ]
Enter a variable list:
(c,b,a)
Enter the number of free variables:
3
Enter a prenex formula:
[ b^2 + a^2 - c^2 <= 0 \/ c (b^2 + a^2 - c^2)^2 < a b^2 (2 a c - (c^2 + a^2 - b^2)) ].
=======================================================

Before Normalization >
assume [ a > 0 /\ b > 0 /\ c > 0 /\ a < b + c /\ b < a + c /\ c < a + b ]

Before Normalization >
finish

An equivalent quantifier-free formula:

a - c < 0

====================================================
\end{verbatim}
}
In other words, given the assumption (i.e. the conditions that <em>a<em>,
<em>b<em> and <em>c<em> really form a triangle), the input formula is equivalent
to <em>a - c < 0<em>.  Simplification has ``discovered'' the theorem
that the external bisector described above exists if and only if side
<em>c<em> is longer than side <em>a<em>.
